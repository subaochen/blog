\hsize=4in % 页宽4in
%\baselineskip 12pt %默认行间距即12pt
%\parskip=0pt%默认值
\everypar{\dag}% 每一段前面加一个宝剑符
\noindent Now let’s see how to control indentation. If an
ordinary word processor can do it, so surely can \TeX. Note
that this paragraph isn’t indented.

Usually you’ll either want to indent paragraphs or to leave
extra space between them. Since we haven’t changed anything
yet, this paragraph is indented.

{\parindent = 30pt \parskip = 6pt
    % The left brace starts a group containing the unindented text.
    Let’s do these three paragraphs a different way,
    with no indentation and six printer’s points of extra space
    between paragraphs.

    So here’s another paragraph that we’re typesetting without
    indentation. If we didn’t put space between these paragraphs,
    you would have a hard time knowing where one ends
    and the next begins.

    The third paragraph should has extra 6pt parskip.
    \par % The paragraph *must* be ended within the group.
}% The right brace ends the group containing unindented text.
%因为上一个段落有\par,因此此处可以没有空行 
It’s also possible to indent both sides of entire paragraphs.
The next three paragraphs illustrate this:
\smallskip % Provide a little extra space here.
% Skips like this and \vskip below end a paragraph.
{\narrower
    ``We've indented this paragraph on both sides by the paragraph
    indentation. This is often a good way to set long quotations.
    You can do multiple paragraphs this way if you choose.''

    ``This is the second paragraph that’s singly indented.''\par}
{\narrower \narrower 
    You can even make paragraphs doubly narrow
    if that’s what you need. This is an example of a doubly
    narrowed paragraph.\par}
\vskip 1pc % Skip down one pica for visual separation.
In this paragraph we’re back to the normal margins, as you can
see for yourself. We’ll let it run on a little longer so that
the margins are clearly visible.

% 空行是必要的
{\leftskip -40pt Now we’ll indent the left margin by half
    an inch and leave the right margin at its usual position.\par}
{\rightskip 20pt Finally, we’ll indent the right margin by half
    an inch and leave the left margin at its usual position.\par}
\bye % end the document

